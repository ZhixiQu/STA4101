% Options for packages loaded elsewhere
% Options for packages loaded elsewhere
\PassOptionsToPackage{unicode}{hyperref}
\PassOptionsToPackage{hyphens}{url}
\PassOptionsToPackage{dvipsnames,svgnames,x11names}{xcolor}
%
\documentclass[
  letterpaper,
  DIV=11,
  numbers=noendperiod]{scrartcl}
\usepackage{xcolor}
\usepackage{amsmath,amssymb}
\setcounter{secnumdepth}{-\maxdimen} % remove section numbering
\usepackage{iftex}
\ifPDFTeX
  \usepackage[T1]{fontenc}
  \usepackage[utf8]{inputenc}
  \usepackage{textcomp} % provide euro and other symbols
\else % if luatex or xetex
  \usepackage{unicode-math} % this also loads fontspec
  \defaultfontfeatures{Scale=MatchLowercase}
  \defaultfontfeatures[\rmfamily]{Ligatures=TeX,Scale=1}
\fi
\usepackage{lmodern}
\ifPDFTeX\else
  % xetex/luatex font selection
\fi
% Use upquote if available, for straight quotes in verbatim environments
\IfFileExists{upquote.sty}{\usepackage{upquote}}{}
\IfFileExists{microtype.sty}{% use microtype if available
  \usepackage[]{microtype}
  \UseMicrotypeSet[protrusion]{basicmath} % disable protrusion for tt fonts
}{}
\makeatletter
\@ifundefined{KOMAClassName}{% if non-KOMA class
  \IfFileExists{parskip.sty}{%
    \usepackage{parskip}
  }{% else
    \setlength{\parindent}{0pt}
    \setlength{\parskip}{6pt plus 2pt minus 1pt}}
}{% if KOMA class
  \KOMAoptions{parskip=half}}
\makeatother
% Make \paragraph and \subparagraph free-standing
\makeatletter
\ifx\paragraph\undefined\else
  \let\oldparagraph\paragraph
  \renewcommand{\paragraph}{
    \@ifstar
      \xxxParagraphStar
      \xxxParagraphNoStar
  }
  \newcommand{\xxxParagraphStar}[1]{\oldparagraph*{#1}\mbox{}}
  \newcommand{\xxxParagraphNoStar}[1]{\oldparagraph{#1}\mbox{}}
\fi
\ifx\subparagraph\undefined\else
  \let\oldsubparagraph\subparagraph
  \renewcommand{\subparagraph}{
    \@ifstar
      \xxxSubParagraphStar
      \xxxSubParagraphNoStar
  }
  \newcommand{\xxxSubParagraphStar}[1]{\oldsubparagraph*{#1}\mbox{}}
  \newcommand{\xxxSubParagraphNoStar}[1]{\oldsubparagraph{#1}\mbox{}}
\fi
\makeatother

\usepackage{color}
\usepackage{fancyvrb}
\newcommand{\VerbBar}{|}
\newcommand{\VERB}{\Verb[commandchars=\\\{\}]}
\DefineVerbatimEnvironment{Highlighting}{Verbatim}{commandchars=\\\{\}}
% Add ',fontsize=\small' for more characters per line
\usepackage{framed}
\definecolor{shadecolor}{RGB}{241,243,245}
\newenvironment{Shaded}{\begin{snugshade}}{\end{snugshade}}
\newcommand{\AlertTok}[1]{\textcolor[rgb]{0.68,0.00,0.00}{#1}}
\newcommand{\AnnotationTok}[1]{\textcolor[rgb]{0.37,0.37,0.37}{#1}}
\newcommand{\AttributeTok}[1]{\textcolor[rgb]{0.40,0.45,0.13}{#1}}
\newcommand{\BaseNTok}[1]{\textcolor[rgb]{0.68,0.00,0.00}{#1}}
\newcommand{\BuiltInTok}[1]{\textcolor[rgb]{0.00,0.23,0.31}{#1}}
\newcommand{\CharTok}[1]{\textcolor[rgb]{0.13,0.47,0.30}{#1}}
\newcommand{\CommentTok}[1]{\textcolor[rgb]{0.37,0.37,0.37}{#1}}
\newcommand{\CommentVarTok}[1]{\textcolor[rgb]{0.37,0.37,0.37}{\textit{#1}}}
\newcommand{\ConstantTok}[1]{\textcolor[rgb]{0.56,0.35,0.01}{#1}}
\newcommand{\ControlFlowTok}[1]{\textcolor[rgb]{0.00,0.23,0.31}{\textbf{#1}}}
\newcommand{\DataTypeTok}[1]{\textcolor[rgb]{0.68,0.00,0.00}{#1}}
\newcommand{\DecValTok}[1]{\textcolor[rgb]{0.68,0.00,0.00}{#1}}
\newcommand{\DocumentationTok}[1]{\textcolor[rgb]{0.37,0.37,0.37}{\textit{#1}}}
\newcommand{\ErrorTok}[1]{\textcolor[rgb]{0.68,0.00,0.00}{#1}}
\newcommand{\ExtensionTok}[1]{\textcolor[rgb]{0.00,0.23,0.31}{#1}}
\newcommand{\FloatTok}[1]{\textcolor[rgb]{0.68,0.00,0.00}{#1}}
\newcommand{\FunctionTok}[1]{\textcolor[rgb]{0.28,0.35,0.67}{#1}}
\newcommand{\ImportTok}[1]{\textcolor[rgb]{0.00,0.46,0.62}{#1}}
\newcommand{\InformationTok}[1]{\textcolor[rgb]{0.37,0.37,0.37}{#1}}
\newcommand{\KeywordTok}[1]{\textcolor[rgb]{0.00,0.23,0.31}{\textbf{#1}}}
\newcommand{\NormalTok}[1]{\textcolor[rgb]{0.00,0.23,0.31}{#1}}
\newcommand{\OperatorTok}[1]{\textcolor[rgb]{0.37,0.37,0.37}{#1}}
\newcommand{\OtherTok}[1]{\textcolor[rgb]{0.00,0.23,0.31}{#1}}
\newcommand{\PreprocessorTok}[1]{\textcolor[rgb]{0.68,0.00,0.00}{#1}}
\newcommand{\RegionMarkerTok}[1]{\textcolor[rgb]{0.00,0.23,0.31}{#1}}
\newcommand{\SpecialCharTok}[1]{\textcolor[rgb]{0.37,0.37,0.37}{#1}}
\newcommand{\SpecialStringTok}[1]{\textcolor[rgb]{0.13,0.47,0.30}{#1}}
\newcommand{\StringTok}[1]{\textcolor[rgb]{0.13,0.47,0.30}{#1}}
\newcommand{\VariableTok}[1]{\textcolor[rgb]{0.07,0.07,0.07}{#1}}
\newcommand{\VerbatimStringTok}[1]{\textcolor[rgb]{0.13,0.47,0.30}{#1}}
\newcommand{\WarningTok}[1]{\textcolor[rgb]{0.37,0.37,0.37}{\textit{#1}}}

\usepackage{longtable,booktabs,array}
\usepackage{calc} % for calculating minipage widths
% Correct order of tables after \paragraph or \subparagraph
\usepackage{etoolbox}
\makeatletter
\patchcmd\longtable{\par}{\if@noskipsec\mbox{}\fi\par}{}{}
\makeatother
% Allow footnotes in longtable head/foot
\IfFileExists{footnotehyper.sty}{\usepackage{footnotehyper}}{\usepackage{footnote}}
\makesavenoteenv{longtable}
\usepackage{graphicx}
\makeatletter
\newsavebox\pandoc@box
\newcommand*\pandocbounded[1]{% scales image to fit in text height/width
  \sbox\pandoc@box{#1}%
  \Gscale@div\@tempa{\textheight}{\dimexpr\ht\pandoc@box+\dp\pandoc@box\relax}%
  \Gscale@div\@tempb{\linewidth}{\wd\pandoc@box}%
  \ifdim\@tempb\p@<\@tempa\p@\let\@tempa\@tempb\fi% select the smaller of both
  \ifdim\@tempa\p@<\p@\scalebox{\@tempa}{\usebox\pandoc@box}%
  \else\usebox{\pandoc@box}%
  \fi%
}
% Set default figure placement to htbp
\def\fps@figure{htbp}
\makeatother





\setlength{\emergencystretch}{3em} % prevent overfull lines

\providecommand{\tightlist}{%
  \setlength{\itemsep}{0pt}\setlength{\parskip}{0pt}}



 


\KOMAoption{captions}{tableheading}
\makeatletter
\@ifpackageloaded{caption}{}{\usepackage{caption}}
\AtBeginDocument{%
\ifdefined\contentsname
  \renewcommand*\contentsname{Table of contents}
\else
  \newcommand\contentsname{Table of contents}
\fi
\ifdefined\listfigurename
  \renewcommand*\listfigurename{List of Figures}
\else
  \newcommand\listfigurename{List of Figures}
\fi
\ifdefined\listtablename
  \renewcommand*\listtablename{List of Tables}
\else
  \newcommand\listtablename{List of Tables}
\fi
\ifdefined\figurename
  \renewcommand*\figurename{Figure}
\else
  \newcommand\figurename{Figure}
\fi
\ifdefined\tablename
  \renewcommand*\tablename{Table}
\else
  \newcommand\tablename{Table}
\fi
}
\@ifpackageloaded{float}{}{\usepackage{float}}
\floatstyle{ruled}
\@ifundefined{c@chapter}{\newfloat{codelisting}{h}{lop}}{\newfloat{codelisting}{h}{lop}[chapter]}
\floatname{codelisting}{Listing}
\newcommand*\listoflistings{\listof{codelisting}{List of Listings}}
\makeatother
\makeatletter
\makeatother
\makeatletter
\@ifpackageloaded{caption}{}{\usepackage{caption}}
\@ifpackageloaded{subcaption}{}{\usepackage{subcaption}}
\makeatother
\usepackage{bookmark}
\IfFileExists{xurl.sty}{\usepackage{xurl}}{} % add URL line breaks if available
\urlstyle{same}
\hypersetup{
  pdftitle={Demonstration},
  pdfauthor={Rohan Alexander},
  colorlinks=true,
  linkcolor={blue},
  filecolor={Maroon},
  citecolor={Blue},
  urlcolor={Blue},
  pdfcreator={LaTeX via pandoc}}


\title{Demonstration}
\author{Rohan Alexander}
\date{8 September, 2025}
\begin{document}
\maketitle


\section{Introduction}\label{introduction}

\texttt{Python} is a general-purpose programming language created by
Guido van Rossum. \texttt{Python} version 0.9.0 was released in February
1991, and the current version, 3.13, was released in October 2024. It
was named \texttt{Python} after \emph{Monty Python's Flying Circus}.

\texttt{Python} is a popular language in machine learning, but it was
designed, and is more commonly used, for more general software
applications. This means that we will especially rely on packages when
we use Python for data science.

\subsection{Python, VS Code, and uv}\label{python-vs-code-and-uv}

There are other options, but the community more broadly has settled on
VS Code (you can pick other options if you have a good reason to do
that). You can download VS Code for free
\href{https://code.visualstudio.com}{here} and then install it.

Open VS Code (Figure~\ref{fig-vscodesetup-a}), and open a new Terminal:
Terminal -\textgreater{} New Terminal (Figure~\ref{fig-vscodesetup-b}).
We can then install \texttt{uv}, which is a Python package manager, by
putting
\texttt{curl\ -LsSf\ https://astral.sh/uv/install.sh\ \textbar{}\ sh}
into the Terminal and pressing ``return/enter'' afterwards
(Figure~\ref{fig-vscodesetup-c}). Finally, to install Python we can use
\texttt{uv} by putting \texttt{uv\ python\ install} into that Terminal
and pressing ``return/enter'' afterwards
(Figure~\ref{fig-vscodesetup-d}).

\begin{figure}

\begin{minipage}{\linewidth}

\centering{

\includegraphics[width=0.5\linewidth,height=\textheight,keepaspectratio]{B-VS_Code-1.png}

}

\subcaption{\label{fig-vscodesetup-a}Opening VS Code}

\end{minipage}%
\newline
\begin{minipage}{\linewidth}

\centering{

\includegraphics[width=0.5\linewidth,height=\textheight,keepaspectratio]{B-VS_Code-2.png}

}

\subcaption{\label{fig-vscodesetup-b}Opening a Terminal in VS Code}

\end{minipage}%
\newline
\begin{minipage}{\linewidth}

\centering{

\includegraphics[width=0.5\linewidth,height=\textheight,keepaspectratio]{B-VS_Code-3.png}

}

\subcaption{\label{fig-vscodesetup-c}Install uv}

\end{minipage}%
\newline
\begin{minipage}{\linewidth}

\centering{

\includegraphics[width=0.5\linewidth,height=\textheight,keepaspectratio]{B-VS_Code-4.png}

}

\subcaption{\label{fig-vscodesetup-d}Install Python}

\end{minipage}%

\caption{\label{fig-vscodesetup}Opening VS Code and a new terminal and
then installing uv and Python}

\end{figure}%

\section{Project set-up}\label{project-set-up}

We are going to get started with an example that downloads some data
from Open Data Toronto. To start, we need to create a project, which
will allow all our code to be self-contained.

Open VS Code and open a new Terminal: ``Terminal'' -\textgreater{} ``New
Terminal''. Then use Unix shell commands to navigate to where you want
to create your folder. For instance, use \texttt{ls} to list all the
folders in the current directory, then move to one using \texttt{cd} and
then the name of the folder. If you need to go back one level then use
\texttt{..}.

Once you are happy with where you are going to create this new folder,
we can use \texttt{uv\ init} in the Terminal to do this, pressing
``return/enter'' afterwards (\texttt{cd} then moves to the new folder
``shelter\_usage'').

\begin{Shaded}
\begin{Highlighting}[]
\NormalTok{\#| eval: false}
\NormalTok{\#| echo: true}

\NormalTok{uv init week\_1}
\NormalTok{cd week\_1}
\end{Highlighting}
\end{Shaded}

By default, there will be a script in the example folder. We want to use
\texttt{uv\ run} to run that script, which will then create an project
environment for us.

\begin{Shaded}
\begin{Highlighting}[]
\NormalTok{\#| eval: false}
\NormalTok{\#| echo: true}

\NormalTok{uv run hello.py}
\end{Highlighting}
\end{Shaded}

A project environment is specific to that project. We will use the
package \texttt{numpy} to simulate data. We need to add this package to
our environment with \texttt{uv\ add}.

\begin{Shaded}
\begin{Highlighting}[]
\NormalTok{\#| eval: false}
\NormalTok{\#| echo: true}

\NormalTok{uv add numpy}
\end{Highlighting}
\end{Shaded}

We can then modify \texttt{hello.py} to use \texttt{numpy} to simulate
from the Normal distribution.

\begin{Shaded}
\begin{Highlighting}[]
\ImportTok{import}\NormalTok{ numpy }\ImportTok{as}\NormalTok{ np}

\KeywordTok{def}\NormalTok{ main():}
\NormalTok{    np.random.seed(}\DecValTok{853}\NormalTok{)}

\NormalTok{    mu, sigma }\OperatorTok{=} \DecValTok{0}\NormalTok{, }\DecValTok{1}
\NormalTok{    sample\_sizes }\OperatorTok{=}\NormalTok{ [}\DecValTok{10}\NormalTok{, }\DecValTok{100}\NormalTok{, }\DecValTok{1000}\NormalTok{, }\DecValTok{10000}\NormalTok{]}
\NormalTok{    differences }\OperatorTok{=}\NormalTok{ []}

    \ControlFlowTok{for}\NormalTok{ size }\KeywordTok{in}\NormalTok{ sample\_sizes:}
\NormalTok{        sample }\OperatorTok{=}\NormalTok{ np.random.normal(mu, sigma, size)}
\NormalTok{        sample\_mean }\OperatorTok{=}\NormalTok{ np.mean(sample)}
\NormalTok{        diff }\OperatorTok{=} \BuiltInTok{abs}\NormalTok{(mu }\OperatorTok{{-}}\NormalTok{ sample\_mean)}
\NormalTok{        differences.append(diff)}
        \BuiltInTok{print}\NormalTok{(}\SpecialStringTok{f"Sample size: }\SpecialCharTok{\{}\NormalTok{size}\SpecialCharTok{\}}\SpecialStringTok{"}\NormalTok{)}
        \BuiltInTok{print}\NormalTok{(}\SpecialStringTok{f"  Difference between sample and population mean: }\SpecialCharTok{\{}\BuiltInTok{round}\NormalTok{(diff, }\DecValTok{3}\NormalTok{)}\SpecialCharTok{\}}\SpecialStringTok{"}\NormalTok{)}
        
\ControlFlowTok{if} \VariableTok{\_\_name\_\_} \OperatorTok{==} \StringTok{"\_\_main\_\_"}\NormalTok{:}
\NormalTok{    main()}
\end{Highlighting}
\end{Shaded}

After we have modified and saved \texttt{hello.py} we can run it with
\texttt{uv\ run} in exactly the same way as before.

At this point we should close VS Code. We want to re-open it to make
sure that our project environment is working as it needs to. In VS Code,
a project is a self-contained folder. You can open a folder with
``File'' -\textgreater{} ``Open Folder\ldots{}'' and then select the
relevant folder, in this case ``week\_1''. You should then be able to
re-run \texttt{uv\ run\ hello.py} and it should work.

\section{Simulate}\label{simulate}

We would like to more thoroughly simulate the dataset that we are
interested in. We will use \texttt{polars} to provide a dataframe to
store our simulated results, so we should add this to our environment
with \texttt{uv\ add}.

\begin{Shaded}
\begin{Highlighting}[]
\NormalTok{\#| eval: false}
\NormalTok{\#| echo: true}

\NormalTok{uv add polars}
\end{Highlighting}
\end{Shaded}

Create a new Python file called \texttt{00-simulate\_data.py}.

\begin{Shaded}
\begin{Highlighting}[]
\CommentTok{\#\#\#\# Preamble \#\#\#\#}
\CommentTok{\# Purpose: Simulates a dataset of daily shelter usage}
\CommentTok{\# Author: Rohan Alexander}
\CommentTok{\# Date: 8 September 2025}
\CommentTok{\# Contact: rohan.alexander@utoronto.ca}
\CommentTok{\# License: MIT}
\CommentTok{\# Pre{-}requisites:}
\CommentTok{\# {-} Add \textasciigrave{}polars\textasciigrave{}: uv add polars}
\CommentTok{\# {-} Add \textasciigrave{}numpy\textasciigrave{}: uv add numpy}
\CommentTok{\# {-} Add \textasciigrave{}datetime\textasciigrave{}: uv add datetime}


\CommentTok{\#\#\#\# Workspace setup \#\#\#\#}
\ImportTok{import}\NormalTok{ polars }\ImportTok{as}\NormalTok{ pl}
\ImportTok{import}\NormalTok{ numpy }\ImportTok{as}\NormalTok{ np}
\ImportTok{from}\NormalTok{ datetime }\ImportTok{import}\NormalTok{ date}

\NormalTok{rng }\OperatorTok{=}\NormalTok{ np.random.default\_rng(seed}\OperatorTok{=}\DecValTok{853}\NormalTok{)}


\CommentTok{\#\#\#\# Simulate data \#\#\#\#}
\CommentTok{\# Simulate 10 shelters and some set capacity}
\NormalTok{shelters\_df }\OperatorTok{=}\NormalTok{ pl.DataFrame(}
\NormalTok{    \{}
        \StringTok{"Shelters"}\NormalTok{: [}\SpecialStringTok{f"Shelter }\SpecialCharTok{\{}\NormalTok{i}\SpecialCharTok{\}}\SpecialStringTok{"} \ControlFlowTok{for}\NormalTok{ i }\KeywordTok{in} \BuiltInTok{range}\NormalTok{(}\DecValTok{1}\NormalTok{, }\DecValTok{11}\NormalTok{)],}
        \StringTok{"Capacity"}\NormalTok{: rng.integers(low}\OperatorTok{=}\DecValTok{10}\NormalTok{, high}\OperatorTok{=}\DecValTok{100}\NormalTok{, size}\OperatorTok{=}\DecValTok{10}\NormalTok{),}
\NormalTok{    \}}
\NormalTok{)}

\CommentTok{\# Create data frame of dates}
\NormalTok{dates }\OperatorTok{=}\NormalTok{ pl.date\_range(}
\NormalTok{    start}\OperatorTok{=}\NormalTok{date(}\DecValTok{2024}\NormalTok{, }\DecValTok{1}\NormalTok{, }\DecValTok{1}\NormalTok{), end}\OperatorTok{=}\NormalTok{date(}\DecValTok{2024}\NormalTok{, }\DecValTok{12}\NormalTok{, }\DecValTok{31}\NormalTok{), interval}\OperatorTok{=}\StringTok{"1d"}\NormalTok{, eager}\OperatorTok{=}\VariableTok{True}
\NormalTok{).alias(}\StringTok{"Dates"}\NormalTok{)}

\CommentTok{\# Convert dates into a data frame}
\NormalTok{dates\_df }\OperatorTok{=}\NormalTok{ pl.DataFrame(dates)}

\CommentTok{\# Combine dates and shelters}
\NormalTok{data }\OperatorTok{=}\NormalTok{ dates\_df.join(shelters\_df, how}\OperatorTok{=}\StringTok{"cross"}\NormalTok{)}

\CommentTok{\# Add usage as a Poisson draw}
\NormalTok{poisson\_draw }\OperatorTok{=}\NormalTok{ rng.poisson(lam}\OperatorTok{=}\NormalTok{data[}\StringTok{"Capacity"}\NormalTok{])}
\NormalTok{usage }\OperatorTok{=}\NormalTok{ np.minimum(poisson\_draw, data[}\StringTok{"Capacity"}\NormalTok{])}

\NormalTok{data }\OperatorTok{=}\NormalTok{ data.with\_columns([pl.Series(}\StringTok{"Usage"}\NormalTok{, usage)])}

\NormalTok{data.write\_parquet(}\StringTok{"simulated\_data.parquet"}\NormalTok{)}
\end{Highlighting}
\end{Shaded}

We can then import our simulated dataset.

\begin{Shaded}
\begin{Highlighting}[]
\ImportTok{import}\NormalTok{ polars }\ImportTok{as}\NormalTok{ pl}

\NormalTok{df }\OperatorTok{=}\NormalTok{ pl.read\_parquet(}\StringTok{"simulated\_data.parquet"}\NormalTok{)}

\BuiltInTok{print}\NormalTok{(df.head(}\DecValTok{5}\NormalTok{))}
\end{Highlighting}
\end{Shaded}

\section{Acquire}\label{acquire}

Use this source:
https://open.toronto.ca/dataset/daily-shelter-overnight-service-occupancy-capacity/

\begin{Shaded}
\begin{Highlighting}[]
\ImportTok{import}\NormalTok{ polars }\ImportTok{as}\NormalTok{ pl}

\CommentTok{\# URL of the CSV file}
\NormalTok{url }\OperatorTok{=} \StringTok{"https://ckan0.cf.opendata.inter.prod{-}toronto.ca/dataset/21c83b32{-}d5a8{-}4106{-}a54f{-}010dbe49f6f2/resource/ffd20867{-}6e3c{-}4074{-}8427{-}d63810edf231/download/Daily}\SpecialCharTok{\%20s}\StringTok{helter}\SpecialCharTok{\%20o}\StringTok{vernight}\SpecialCharTok{\%20o}\StringTok{ccupancy.csv"}

\CommentTok{\# Read the CSV file into a Polars DataFrame}
\NormalTok{df }\OperatorTok{=}\NormalTok{ pl.read\_csv(url)}

\CommentTok{\# Save the raw data}
\NormalTok{df.write\_parquet(}\StringTok{"shelter\_usage.parquet"}\NormalTok{)}
\end{Highlighting}
\end{Shaded}

We are likely only interested in a few columns and only rows where there
are data.

\begin{Shaded}
\begin{Highlighting}[]
\ImportTok{import}\NormalTok{ polars }\ImportTok{as}\NormalTok{ pl}

\NormalTok{df }\OperatorTok{=}\NormalTok{ pl.read\_parquet(}\StringTok{"shelter\_usage.parquet"}\NormalTok{)}

\CommentTok{\# Select specific columns}
\NormalTok{selected\_columns }\OperatorTok{=}\NormalTok{ [}\StringTok{"OCCUPANCY\_DATE"}\NormalTok{, }\StringTok{"SHELTER\_ID"}\NormalTok{, }\StringTok{"OCCUPIED\_BEDS"}\NormalTok{, }\StringTok{"CAPACITY\_ACTUAL\_BED"}\NormalTok{]}

\NormalTok{selected\_df }\OperatorTok{=}\NormalTok{ df.select(selected\_columns)}

\CommentTok{\# Filter to only rows that have data}
\NormalTok{filtered\_df }\OperatorTok{=}\NormalTok{ selected\_df.}\BuiltInTok{filter}\NormalTok{(df[}\StringTok{"OCCUPIED\_BEDS"}\NormalTok{].is\_not\_null())}

\BuiltInTok{print}\NormalTok{(filtered\_df.head())}

\NormalTok{renamed\_df }\OperatorTok{=}\NormalTok{ filtered\_df.rename(\{}\StringTok{"OCCUPANCY\_DATE"}\NormalTok{: }\StringTok{"date"}\NormalTok{,}
                                 \StringTok{"SHELTER\_ID"}\NormalTok{: }\StringTok{"Shelters"}\NormalTok{,}
                                 \StringTok{"CAPACITY\_ACTUAL\_BED"}\NormalTok{: }\StringTok{"Capacity"}\NormalTok{,}
                                 \StringTok{"OCCUPIED\_BEDS"}\NormalTok{: }\StringTok{"Usage"}
\NormalTok{                                 \})}

\BuiltInTok{print}\NormalTok{(renamed\_df.head())}

\NormalTok{renamed\_df.write\_parquet(}\StringTok{"cleaned\_shelter\_usage.parquet"}\NormalTok{)}
\end{Highlighting}
\end{Shaded}

\section{Explore}\label{explore}

Manipulate the data into a summary table.

\begin{Shaded}
\begin{Highlighting}[]
\ImportTok{import}\NormalTok{ polars }\ImportTok{as}\NormalTok{ pl}

\NormalTok{df }\OperatorTok{=}\NormalTok{ pl.read\_parquet(}\StringTok{"cleaned\_shelter\_usage.parquet"}\NormalTok{)}

\CommentTok{\# Convert the date column to datetime and rename it for clarity}
\NormalTok{df }\OperatorTok{=}\NormalTok{ df.with\_columns(pl.col(}\StringTok{"date"}\NormalTok{).}\BuiltInTok{str}\NormalTok{.strptime(pl.Date, }\StringTok{"\%Y{-}\%m{-}}\SpecialCharTok{\%d}\StringTok{"}\NormalTok{).alias(}\StringTok{"date"}\NormalTok{))}

\CommentTok{\# Group by "Dates" and calculate total "Capacity" and "Usage"}
\NormalTok{aggregated\_df }\OperatorTok{=}\NormalTok{ (}
\NormalTok{    df.group\_by(}\StringTok{"date"}\NormalTok{)}
\NormalTok{    .agg([}
\NormalTok{        pl.col(}\StringTok{"Capacity"}\NormalTok{).}\BuiltInTok{sum}\NormalTok{().alias(}\StringTok{"Total\_Capacity"}\NormalTok{),}
\NormalTok{        pl.col(}\StringTok{"Usage"}\NormalTok{).}\BuiltInTok{sum}\NormalTok{().alias(}\StringTok{"Total\_Usage"}\NormalTok{)}
\NormalTok{    ])}
\NormalTok{    .sort(}\StringTok{"date"}\NormalTok{)  }\CommentTok{\# Sort the results by date}
\NormalTok{)}

\CommentTok{\# Display the aggregated DataFrame}
\BuiltInTok{print}\NormalTok{(aggregated\_df)}
\end{Highlighting}
\end{Shaded}





\end{document}
